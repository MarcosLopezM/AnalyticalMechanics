\documentclass[../main.tex]{subfiles}

\begin{document}
\begin{problema}
	Un oscilador sobreamortiguado satisface la ecuación

	\begin{equation*}
		\ddot{x} + 10\dot{x} + 16x = 0.
	\end{equation*}

	En \(t = 0\) la partícula está en \(x = 1\) y sale con
	dirección hacia el origen con velocidad \(u\).
	Encuentra la trayectoria de dicha partícula.
	Muestra además que la partícula llegará al origen de
	coordenadas si

	\begin{equation}
		\dfrac{u - 2}{u - 8} = \mathrm{e}^{6t}.
		\label{eq:condition}
	\end{equation}

	¿Qué tan grande tiene que ser \(u\) para que la
	partícula se pase del origen de coordenadas?
\end{problema}

\startsolution

Para encontrar la solución general a la ecuación diferencial
proponemos \(x = \mathrm{e}^{\lambda t}\), tal que,

\begin{align*}
	\dot{x}  & = \lambda \mathrm{e}^{\lambda t},    \\
	\ddot{x} & = \lambda^{2} \mathrm{e}^{\lambda t}
\end{align*}

Sustituyendo las expresiones anteriores tenemos que el polinomio
característico queda de la forma:

\begin{align*}
	\lambda^{2} + 10\lambda + 16 & = 0,                          \\
	\Longrightarrow \lambda_{1}  & = -2 \wedge \lambda_{2} = -8.
\end{align*}

De esta manera, la solución y su derivada quedan como:

\begin{align}
	x(t)       & = A \mathrm{e}^{-2t} + B \mathrm{e}^{-8t},\label{eq:sol-pos}    \\
	\dot{x}(t) & = -2A \mathrm{e}^{-2t} - 8B \mathrm{e}^{-8t}.\label{eq:sol-vel}
\end{align}

Sustituyendo las condiciones iniciales \(x(0) = 1\) y \(x(0) = -u\),
respectivamente, en las \zcref{eq:sol-pos,eq:sol-vel},

\begin{align}
	x(0) = 1        & = A \mathrm{e}^{0} + B \mathrm{e}^{0},\nonumber    \\
	A + B           & = 1.\label{eq:first-syst}                          \\
	\dot{x}(t) = -u & = -2A \mathrm{e}^{0} - 8B \mathrm{e}^{0},\nonumber \\
	2A + 8B         & = u.\label{eq:second-syst}
\end{align}

Multiplicando la \zcref{eq:first-syst} por 2 y
restándola a la \zcref{eq:second-syst},

\begin{align}
	6B           & = u - 2,\nonumber                          \\
	\Aboxedsec{B & = \dfrac{u - 2}{6}.}\label{eq:second-coef}
\end{align}

Sustituyendo la \zcref{eq:second-coef} en la \zcref{eq:first-syst},

\begin{align}
	A + \dfrac{u - 2}{6} & = 1,\nonumber                                \\
	A                    & = \dfrac{6 - u + 2}{6},\nonumber             \\,
	\Aboxedsec{A         & = \dfrac{8 - u}{6}.}\label{eq:eq:first-coef}
\end{align}

Sustituyendo las \zcref{eq:eq:first-coef, eq:second-coef} en
la \zcref{eq:sol-pos} obtenemos que la solución general es

\begin{empheq}[box = \mainresult]{equation}
	x(t) = \dfrac{8 - u}{6}\mathrm{e}^{-2t} + \dfrac{u - 2}{6}\mathrm{e}^{-8t}.
	\label{eq:gral-sol-p3}
\end{empheq}

Para determinar que la partícula llegará al origen si se satisface la
\zcref{eq:condition}, igualamos la \zcref{eq:gral-sol-p3} a cero,
i.e.

\begin{align}
	\dfrac{8 - u}{6}\mathrm{e}^{-2t} + \dfrac{u - 2}{6}\mathrm{e}^{-8t} & = 0,\nonumber                          \\
	(8 - u)\mathrm{e}^{-2t} + (u - 2)\mathrm{e}^{-8t}                   & = 0,\nonumber                          \\
	(8 - u) + (u - 2)\mathrm{e}^{-6t}                                   & = 0,\nonumber                          \\
	(u - 2)\mathrm{e}^{-6t}                                             & = -(8 - u),\nonumber                   \\
	\Aboxedmain{\dfrac{u - 2}{u - 8}                                    & = \mathrm{e}^{6t}.}\label{eq:sat-cond}
\end{align}

Para determinar el valor máximo de la velocidad que nos asegure que la
partícula se pase del origen de coordenadas, usamos la prueba de la segunda
derivada. Primero resolvemos la \zcref{eq:sat-cond} para \(u\),

\begin{align}
	u - 2                  & = u \mathrm{e}^{6t} - 8\mathrm{e}^{6t},\nonumber      \\
	u(1 - \mathrm{e}^{6t}) & = 2 - 8\mathrm{e}^{6t},\nonumber                      \\
	\Aboxedsec{u           & = \dfrac{2 - 8\mathrm{e}^{6t}}{1 - \mathrm{e}^{6t}}.}
\end{align}
\end{document}
