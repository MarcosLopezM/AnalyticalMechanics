\documentclass[../main.tex]{subfiles}

\begin{document}
\begin{problema}
	Un oscilador sobreamortiguado satisface la ecuación

	\begin{equation}
		\ddot{x} + 10\dot{x} + 16x = 0.
	\end{equation}

	En \(t = 0\) la partícula está en \(x = 1\) y sale con
	dirección hacia el origen con velocidad \(u\).
	Encuentra la trayectoria de dicha partícula.
	Muestra además que la partícula llegará al origen de
	coordenadas si

	\begin{equation}
		\dfrac{u - 2}{u - 8} = \mathrm{e}^{6t}.
	\end{equation}

	¿Qué tan grande tiene que ser \(u\) para que la
	partícula se pase del origen de coordenadas?
\end{problema}
\end{document}
