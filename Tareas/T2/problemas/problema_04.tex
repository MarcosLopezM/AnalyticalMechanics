\documentclass[../main.tex]{subfiles}

\begin{document}
\begin{problema}
	Una partícula de masa \(m\), se mueve sobre el
	eje negativo \(x\) hacia el origen de coordenadas
	con velocidad constante \(u\). Cuando la partícula
	llega al origen siente una fuerza \(F = -kx^{2}\),
	con \(k\) una constante positiva. Calcula la
	distancia máxima que avanza la partícula sobre el
	positivo \(x\).
\end{problema}

\startsolution

Por hipótesis sabemos que la partícula está bajo la influencia
de una fuerza conservativa, que depende únicamente de la
posición; entonces por la segunda ley de Newton tenemos que

\begin{align}
	F(x)                                             & = mv\odv{v}{x},\nonumber                                                                                 \\
	\int_{x_{0}}^{x} F(x^{\prime}) \odif{x^{\prime}} & = m\int_{v_{0}}^{v} v^{\prime} \odif{v^{\prime}} = \dfrac{1}{2}mv^{2} - \dfrac{1}{2}mv_{0}^{2},\nonumber \\
	\Longrightarrow\; E                              & = -\dfrac{1}{2}mu^{2} - \int_{x_{0}}^{x} F(x^{\prime}) \odif{x^{\prime}},\label{eq:energy}
\end{align}

donde \(E = -\tfrac{1}{2}mu^{2}\). Y, puesto que la energía mecánica total del sistema
es igual a \( E = K + V\), tenemos que la energía potencial del sistema
está dada como

\begin{align}
	V(x) & = -\int F(x^{\prime}) \odif{x^{\prime}},\nonumber                \\
	     & = - \int_{-x}^{0} -k{x^{\prime}}^{2} \odif{x^{\prime}},\nonumber \\
	V(x) & = \dfrac{k}{3}x^{3}.\label{eq:PotentialEnergy}
\end{align}

Sustituyendo la \zcref{eq:PotentialEnergy} en la \zcref{eq:energy} y,
como deseamos conocer la distancia máxima que avanza la partícula, igualamos la expresión
a cero y resolvemos para \(x\),

\begin{align*}
	-\dfrac{1}{2}mu^{2} + \dfrac{k}{3}x^{3} & = 0,                                        \\
	x^{3}                                   & = \dfrac{3m}{2k}u^{2},                      \\
	\Aboxedmain{x                           & = \Biggl(\dfrac{3m}{2k}u^{2}\Biggr)^{1/3}.}
\end{align*}
\end{document}
