\documentclass[../main.tex]{subfiles}

\begin{document}
\begin{problema}
	Un oscilador forzado satisface la siguiente ecuación

	\begin{equation}
		\ddot{x} + \Omega^{2}x = F_{0}\cos \biggl[\Omega(1 + \epsilon)t\biggr]
	\end{equation}

	con \(\epsilon\) una constante positiva. Muestra que la solución
	que satisface las condiciones iniciales \(x = 0\) y
	\(\dot{x} = 0\) cuando \(t = 0\) es

	\begin{equation}
		x = \dfrac{F_{0}}{\epsilon \Bigl(1 + \tfrac{1}{2}\Bigr)\Omega^{2}}\sin \Bigl[\tfrac{1}{2}\epsilon\Omega t\Bigr]\sin \Bigl[\Omega(1 + \epsilon)t\Bigr].
	\end{equation}

	Grafica \(x(t)\) cuando \(\epsilon\) es pequeña.
\end{problema}

\startsolution
\end{document}
