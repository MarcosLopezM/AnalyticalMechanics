\documentclass[../main.tex]{subfiles}

\begin{document}
\begin{problema}
	El desplazamiento de un objeto con masa \(m\) sujeta a un resorte
	bajo la acción de la fricción de Coulomb satisface la siguiente
	ecuación,

	\begin{equation}
		\ddot{x} + \Omega^{2}x =
		\begin{dcases}
			-F_{0} & \dot{x} > 0 \\
			F_{0}  & \dot{x} < 0 \\
		\end{dcases}
	\end{equation}

	con \(\Omega\) y \(F_{0}\) constantes positivas. Si \(\abs{x} > F_{0}/\Omega^{2}\)
	cuando \(\dot{x} = 0\), el objeto continúa en movimiento mientras
	que si \(\abs{x} \leq F_{0}/\Omega^{2}\) cuando \(\dot{x} = 0\),
	entonces el objeto se detiene. Inicialmente el cuerpo empieza
	en reposo con \(x = 9F_{0}/2\Omega^{2}\). Encuentra donde se detiene
	la masa y por cuanto tiempo estuvo en movimiento.
\end{problema}

\startsolution

Las dos encontrar solución son

\begin{align*}
	\ddot{x} + \Omega^{2} & = -F_{0}, \txt{para} \dot{x} > 0, \\
	\ddot{x} + \Omega^{2} & = F_{0}. \txt{para} \dot{x} < 0.
\end{align*}

Sabemos que su solución general es de la forma

\begin{equation*}
	x(t) = x_{H}(t) + x_{P}(t),
\end{equation*}

donde \(x_{H}\) es la solución a la ecuación homogénea y \(x_{P}\) la solución
a la ecuación particular.

Primero resolvemos la ecuación diferencial homogénea, que es la misma para ambas
ecuaciones,

\begin{equation*}
	\ddot{x} + \Omega^{2}x = 0.
\end{equation*}

Inmediatamente se puede ver que su solución es de la forma

\begin{equation*}
	x(t) = A\cos(\Omega t) + B\sin(\Omega t).
\end{equation*}

Para resolver \(x_{P}(t)\) proponemos una solución de la forma

\begin{equation*}
	x_{P}(t) = C,
\end{equation*}

tal que

\begin{equation*}
	\dot{x}_{P} = 0 = \ddot{x}_{P}.
\end{equation*}

Entonces,

\begin{align*}
	C_{1} & = -\dfrac{F_{0}}{\Omega^{2}}, \txt{para} \dot{x} > 0 \\
	C_{2} & = \dfrac{F_{0}}{\Omega^{2}}. \txt{para} \dot{x} < 0
\end{align*}

Por lo que las soluciones generales,

\begin{align}
	x_{1}(t) & = A\cos(\Omega t) + B\sin(\Omega t) - \dfrac{F_{0}}{\Omega^{2}}, \txt{para} \dot{x} > 0, \label{eq:sol-greater-coef} \\
	x_{2}(t) & = C\cos(\Omega t) + D\sin(\Omega t) + \dfrac{F_{0}}{\Omega^{2}}, \txt{para} \dot{x} < 0, \label{eq:sol-less-coef}
\end{align}

Queremos conocer cómo es el movimiento después de \(t = 0\), por lo que
nos fijamos en la fuerza restaurativa, i.e. \(F_{R} = -\Omega^{2}x\).
Sabemos que \(x(0)\)

\begin{align*}
	F_{R} & = -\Omega^{2}\biggl(\dfrac{9F_{0}}{2\Omega^{2}}\biggr), \\
	F_{R} & = -\dfrac{9F_{0}}{2}.
\end{align*}

De las relaciones que se nos dan, la magnitud de la fuerza restaurativa
es mayor que \(F_{0}\), i.e.

\begin{equation*}
	\dfrac{9F_{0}}{2} > F_{0}.
\end{equation*}

Es decir, el objeto continúa en movimiento y, en particular como la fuerza
restaurativa es mayor en magnitud y su dirección es contraria al movimiento,
usamos \zcref{eq:sol-less-coef}.

\begin{align*}
	x(0) = \dfrac{9F_{0}}{2\Omega^{2}} & = A\cos(\Omega t) + B\sin(\Omega t) + \dfrac{F_{0}}{\Omega^{2}}, \nonumber \\
	                                   & = A\cos(0) + B\sin(0) + \dfrac{F_{0}}{\Omega^{2}},              \nonumber  \\
	\Longrightarrow\; A                & = \dfrac{9F_{0}}{2\Omega^{2}} - \dfrac{F_{0}}{\Omega^{2}},       \nonumber \\
	\Aboxedsec{A                       & = \dfrac{7F_{0}}{2\Omega^{2}}.}                                            \\
	\dot{x}(0) = 0                     & = -A\Omega\sin(0) + B\Omega\cos(0),                                        \\
	\Aboxedsec{B                       & = 0.}
\end{align*}

Por lo que solución es

\begin{equation}
	x(t) = \dfrac{7F_{0}}{2\Omega^{2}}\cos(\Omega t) + \dfrac{F_{0}}{\Omega^{2}}.
	\label{eq:sol-less-gral}
\end{equation}

Ahora para encontrar el tiempo que tardó en detenerse
estudiamos \(\dot{x}(t) = 0\),

\begin{align*}
	\dot{x}(t)     & = -\dfrac{7F_{0}}{2\Omega^{2}}\Omega\sin(\Omega t) = 0, \\
	\sin(\Omega t) & = 0.
\end{align*}

Sabemos que los ceros de la función seno están en

\begin{align*}
	\Omega t & = n\pi,                                 \\
	t        & = \dfrac{n\pi}{\Omega}, n\in\mathbb{N}.
\end{align*}

En particular, como queremos saber que pasa después, consideramos \(n = 1\),

\begin{equation}
	t = \dfrac{\pi}{\Omega}.
	\label{eq:t-1st-left}
\end{equation}

Sustituyendo la \zcref{eq:t-1st-left} en la \zcref{eq:sol-less-gral}
para determinar la distancia a la que se detiene,

\begin{align*}
	x \biggl(\dfrac{\pi}{\Omega}\biggr)           & = \dfrac{7F_{0}}{2\Omega^{2}}\cos \biggl[\Omega\biggl(\dfrac{\pi}{\Omega}\biggr)\biggr] + \dfrac{F_{0}}{\Omega^{2}}, \\
	                                              & = -\dfrac{7F_{0}}{2\Omega^{2}} + \dfrac{F_{0}}{\Omega^{2}},                                                          \\
	\Aboxedsec{x\biggl(\dfrac{\pi}{\Omega}\biggr) & = \dfrac{-5F_{0}}{2\Omega^{2}}.}
\end{align*}

Sabiendo que la fuerza restaurativa está dada como
\(F_{R} = \Omega^{2}x\),

\begin{align*}
	F_{R} & = \Omega^{2}x \biggl(\dfrac{\pi}{\Omega}\biggr),        \\
	      & = \Omega^{3}\biggl(\dfrac{-5F_{0}}{2\Omega^{2}}\biggr), \\
	F_{R} & = \dfrac{-5F_{0}}{2}.
\end{align*}

Por lo tanto, \( \tfrac{5F_{0}}{2} > F_{0}\). Si ahora sustituimos
las condiciones iniciales \( x \bigl(\tfrac{2\pi}{\Omega}\bigr) = \tfrac{-5F_{0}}{2\Omega^{2}}\) y \(\dot{x}\bigl(\tfrac{2\pi}{\Omega}\bigr) = 0\) en
la \zcref{ eq:sol-greater-coef},

\begin{align*}
	x\bigl(\tfrac{2\pi}{\Omega}\bigr) = \dfrac{-5F_{0}}{2\Omega^{2}} & = C\cos(2\pi) + D\sin(2\pi) - \dfrac{F_{0}}{\Omega^{2}},    \\
	\dfrac{-5F_{0}}{2\Omega^{2}}                                     & = C - \dfrac{F_{0}}{\Omega^{2}},                            \\
	\Longrightarrow C                                                & = \dfrac{-5F_{0}\Omega^{2} - 2F_{0}\Omega^{2}}{2\Omega^{4}} \\
	\Aboxedsec{C                                                     & = \dfrac{-3F_{0}}{2\Omega^{2}}.}                            \\
	\dot{x}\bigl(\tfrac{2\pi}{\Omega}\bigr) = 0                      & = -\Omega C\sin(2\pi) + D\cos(2\pi),                        \\
	0                                                                & = D\Omega,                                                  \\
	\Aboxedsec{D                                                     & = 0.}
\end{align*}

Por lo tanto, la solución general para este caso queda como:

\begin{equation*}
	x(t) = \dfrac{-3F_{0}}{2\Omega^{2}}\cos(\Omega t) - \dfrac{F_{0}}{\Omega^{2}}.
\end{equation*}

Finalmente encontramos el tiempo al que se detiene:

\begin{align*}
	\dot{x}(t)        & = -\dfrac{3F_{0}}{2\Omega^{2}}\Omega\sin(\Omega t) = 0, \\
	\sin(\Omega t)    & = 0,                                                    \\
	\Longrightarrow t & = \dfrac{n\pi}{\Omega}.
\end{align*}

Ahora con \(n = 3\), por lo que la distancia a \(t = \tfrac{3\pi}{\Omega}\) es,

\begin{align*}
	x(t) & = -\dfrac{3F_{0}}{2\Omega^{2}}\cos(3\pi),                                               \\
	     & = \dfrac{3F_{0}}{2\Omega^{2}} - \dfrac{F_{0}}{\Omega^{2}} = \dfrac{F_{0}}{2\Omega^{2}}.
\end{align*}

Si calculamos nuevamente la fuerza restaurativa:

\begin{align*}
	F_{R} & = -\Omega^{2}\biggl(\dfrac{F_{0}}{2\Omega^{2}}\biggr), \\
	      & = -\dfrac{F_{0}}{2},                                   \\
	\therefore\; \dfrac{F_{0}}{2} < F_{0}.
\end{align*}

Por lo tanto, el cuerpo se detiene después de completar un período en la
posición \(\tfrac{F_{0}}{2\Omega^{2}}\).
\end{document}
