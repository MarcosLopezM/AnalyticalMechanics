\documentclass[../main.tex]{subfiles}

\begin{document}
\begin{problema}
	The Lagrangian for a relativistic point particle, of mass \(m\), is

	\begin{equation*}
		L = -mc^{2}\sqrt{1 - (\boldvect\vec{\dot{r}}\cdot \boldvect\vec{\dot{r}})/c^{2}} - U(\boldvect\vect{r})
	\end{equation*}

	where \(c\) is the speed of light. Derive the equation of motion, and
	show that it reduces to Newton's equation of motion in the limit
	\( \abs{\boldvect\vect{\dot{r}}} \lll c\).
\end{problema}

\startsolution

Obtenemos las ecuaciones de Euler-Lagrange,

\begin{align*}
	\pdv{L}{\dot{\bm{r}}} & = -mc^{2}\pdv*{\Biggl[\sqrt{1 - \dfrac{\dot{\bm{r}}^{2}}{c^{2}}} - U(\bm{r})\Biggr]}{\dot{\bm{r}}},\nonumber                             \\
	                      & = -mc^{2}\dfrac{\tfrac{1}{2}\Bigl(-\tfrac{2}{c^{2}}\dot{\bm{r}}\Bigr)}{\sqrt{1 - \dfrac{\dot{\bm{r}}^{2}}{c^{2}}} - U(\bm{r})},\nonumber \\
	\odv{L}{\dot{\bm{r}}} & = \dfrac{m \dot{\bm{r}}}{\sqrt{1 - \dfrac{\dot{\bm{r}}^{2}}{c^{2}}}}.
\end{align*}

En el límite \(\abs{\dot{\bm{r}}} \lll c\), la expresión anterior se reduce a

\begin{equation*}
	\pdv{L}{\dot{\bm{r}}} = m \dot{\bm{r}}.
\end{equation*}

Por lo que las ecuaciones de Euler-Lagrange son

\begin{equation*}
	\odv{(m \dot{\bm{r}})}{t} = \pdv{U(r)}{\bm{r}},
\end{equation*}

pero \(\bm{p} = m \dot{\bm{r}}\), pero

\begin{align*}
	\odv{\bm{p}}{t}    & = \pdv{U(\bm{r})}{\bm{r}} = \bm{F}, \\
	\Aboxedmain{\bm{F} & = \odv{\bm{p}}{t}.}
\end{align*}

Por lo tanto, recuperamos la expresión de la segunda ley de Newton.
\end{document}
