\documentclass[../main.tex]{subfiles}

\begin{document}
\begin{problema}
	A bead slides along a smooth wire that has the shape of a parabola
	\(z = cr^{2}\). At equilibrium, the bead rotates in a circle of radius \(R\)
	when the wire is rotating about its vertical symmetry axis with angular velocity
	\(\omega\). Find the value of \(c\).

	\startsolution

	Usando coordenadas cilíndricas debido a la simetría del prolema, tenemos que

	\begin{align*}
		T & = \dfrac{1}{2}m \dot{r}^{2} + \dfrac{1}{2}m \dot{z}^{2} + \dfrac{1}{2}m(r \dot{\theta})^{2}, \\
		U & = mgz,
	\end{align*}

	tal que en \(z = 0 \implies U = 0\).

	Usando la constricción que nos dice que la cuenta se desliza sobre la parábola
	podemos escribir el Lagrangiano en función de \(r\),

	\begin{align*}
		z       & = cr^{2},      \\
		\dot{z} & = 2cr \dot{r}.
	\end{align*}

	Además, recordamos que la velocidad angular depende del tiempo, i.e.

	\begin{align*}
		\theta       & = \omega t, \\
		\dot{\theta} & = \theta.
	\end{align*}

	A partir de las expresiones anteriores, podemos escribir el Lagrangiano que
	describe la dinámica del sistema como

	\begin{equation*}
		L  = \dfrac{1}{2}m(\dot{r}^{2} + 4c^{2}r^{2} \dot{r} + r^{2}\omega^{2}) - mgcr^{2}.
	\end{equation*}

	Obtenemos la ecuación de movimiento a paritr de la ecuación de Euler-Lagrange,

	\begin{align*}
		\odv*{\Biggl[\dfrac{1}{2}m \Bigl(2 \dot{r} + 8c^{2}r^{2} \dot{r}\Bigr)\Biggr]}{t} - \dfrac{1}{2}m \Bigl(8c^{2}r \dot{r}^{2} + 2\theta^{2}r\Bigr) + 2mgcr & = 0, \\
		m \ddot{r} + 8mc^{2}r \dot{r}^{2} + 4mc^{2}r^{2}\ddot{r} - 4mc^{2}r \dot{r}^{2} - m\omega^{2}r + 2mgcr                                                   & = 0, \\
		(1 + 4c^{2}r^{2})\ddot{r} + (4mc^{2}r)\dot{r}^{2} + (2gc - \omega^{2})r                                                                                  & = 0.
	\end{align*}

	Y debido a que la partícula rota en un círculo de radio \(r = R = cte\), la
	ecuación de movimiento se reduce a

	\begin{equation*}
		(2gc - \omega^{2})R = 0.
	\end{equation*}

	Finalmente, resolver la expresión anterior para determinar el valor de \(c\),

	\begin{align*}
		2gc           & = \omega^{2},              \\
		gc            & = \dfrac{\omega^{2}}{2},   \\
		\Aboxedmain{c & = \dfrac{\omega^{2}}{2g}.}
	\end{align*}
\end{problema}
\end{document}
