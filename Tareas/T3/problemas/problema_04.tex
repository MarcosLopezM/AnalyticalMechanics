\documentclass[../main.tex]{subfiles}

\begin{document}
\begin{problema}
	Considera una familia de órbitas en un potencial central
	para el cual la energía total es constante. Muestra que si
	existe una órbita circular estable, entonces el momento angular
	asociado a dicha órbita es más grande que el valor del momento
	angular de cualquier otra órbita de la familia.
\end{problema}

\startsolution

Partimos de la conservación de la energía:

\begin{equation*}
	E = \dfrac{1}{2}m \dot{r}^{2} + \dfrac{L^{2}}{2mr^{2}} + U(r).
\end{equation*}

Escribiendo \(L^{2}\) en términos de la energía,

\begin{equation*}
	L^{2} = 2mr^{2}\Biggl(E - U(r) - \tfrac{1}{2}m \dot{r}^{2}\Biggr).
\end{equation*}

Sabemos que para que una órbita sea circular en \(r = r_{0}\)
sea estable si \(\dot{r} = 0\) para todo tiempo \(t\),
lo cual es posible si \(\pdv{U_{eff}}{r} = 0\). Entonces,

\begin{align*}
	\Bigl. \pdv{U_{eff}}{r}\Bigr\rvert_{r = r_{0}} & = -\dfrac{L^{2}}{2mr^{2}} + U^{\prime}(r) = 0 , \\
	\Rightarrow\; \dfrac{L^{2}}{2mr_{0}^{3}}       & = U^{\prime}(r_{0}),                            \\
	L_{r_{0}}^{2}                                  & = 2mr_{0}^{3}U^{\prime}(r_{0}).
\end{align*}

Por lo tanto,

\begin{align*}
	L^{2} < L_{r_{0}}^{2}, \\
	L < L_{r_{0}}.
\end{align*}

Es decir el momento angular para la órbita circular es más
grande que el de cualquier otra órbita de la familia.
\end{document}
