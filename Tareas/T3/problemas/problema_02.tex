\documentclass[../main.tex]{subfiles}

\begin{document}
\begin{problema}
	Describe como es el movimiento de una partícula sujeta a
	una fuerza repulsiva dada por

	\begin{equation}
		F(r) = kr.
	\end{equation}

	Muestra que la órbita de la misma solo puede ser hiperbólica.
\end{problema}

\startsolution

Obtenemos el potencial,

\begin{align}
	U(r) & = - \int \odif[sep-end=\medspace]{r} kr ,\nonumber \\
	U(r) & = -\dfrac{kr^{2}}{2}.\label{eq:Potential-P2}
\end{align}

Por lo que el potencial efectivo es

\begin{equation}
	U^{eff}(r) = \dfrac{L^{2}}{2mr^{2}} - \dfrac{kr^{2}}{2}.
	\label{eq:EffectivePotential-P2}
\end{equation}

La energía total del sistema es

\begin{align*}
	E                        & = \dfrac{1}{2}m\dot{r}^{2} + \dfrac{L^{2}}{2mr^{2}} - \dfrac{kr^{2}}{2},\nonumber                                     \\
	\dfrac{1}{2}m\dot{r}^{2} & = E - \dfrac{L^{2}}{2mr^{2}} + \dfrac{kr^{2}}{2},\nonumber                                                            \\
	\odv{r}{t}               & = \sqrt{\dfrac{2}{m}}\sqrt{E - \dfrac{L^{2}}{2mr^{2}} + \dfrac{kr^{2}}{2}},                                           \\
	\theta(r)                & = \int \odif[sep-end=\medspace]{r} \dfrac{\tfrac{L}{r^{2}}}{2m\sqrt{E - \tfrac{L^{2}}{2mr^{2}} + \tfrac{kr^{2}}{2}}}.
\end{align*}

Hacemos el siguiente cambio de variable para resolver la integral

\begin{alignat*}{3}
	x        & = r^{2}       & {}\Rightarrow{} & r                    & = \sqrt{x}, \\
	\odif{x} & = 2r \odif{r} & {}\Rightarrow{} & \dfrac{\odif{x}}{2r} & = \odif{r}.
\end{alignat*}

Por lo que integral queda como:

\begin{align*}
	\theta(x) & = \int \odif[sep-end=\medspace]{x} \dfrac{1}{x\sqrt{4 \Biggl(\tfrac{2Emx}{L^{2}} + \tfrac{kmx^{2}}{L^{2}} - 1\Biggr)}}, \\
	          & = \dfrac{1}{2}\int \odif[sep-end=\medspace]{x} \dfrac{1}{x\sqrt{\tfrac{km}{L2}x^{2} + \tfrac{2Em}{L^{2}}x - 1}}.
\end{align*}

Por lo que,

\begin{align*}
	\theta(x) & = \dfrac{1}{2}\arcsin \Biggl[\dfrac{\bigl(\tfrac{2Em}{L^{2}}x\bigr) - 2}{\abs{x}\sqrt{\biggl(\tfrac{2Em}{L^{2}}\biggr)^{2} + 4 \Biggl(\tfrac{km}{L^{2}}\Biggr)}}\Biggr] + \theta_{0}, \\
	          & = \dfrac{1}{2} \arcsin \Biggl[\dfrac{\tfrac{Em}{L^{2}}r^{2} - 1}{r^{2}\sqrt{\tfrac{E^{2}m^{2}}{L^{4}} + \tfrac{km}{L^{2}}}}\Biggr] + \theta_{0}.
\end{align*}

O bien,

\begin{align*}
	\sin(2(\theta - \theta_{0})) & = \dfrac{\tfrac{Em}{L^{2}}r^{2} - 1}{\tfrac{Em}{L^{2}}r^{2}\sqrt{1 + \tfrac{kL^{2}}{mE}}},                  \\
	\sin(2(\theta - \theta_{0})) & = \tfrac{1}{\sqrt{1 + \tfrac{kL^{2}}{mE}}} - \dfrac{\tfrac{L^{2}}{Em}}{r^{2}\sqrt{1 + \tfrac{kL^{2}}{mE}}}.
\end{align*}

Definimos las siguiente constantes:

\begin{align}
	\alpha      & = \dfrac{L^{2}}{Em},             \\
	\varepsilon & = \sqrt{1 + \tfrac{kL^{2}}{mE}}.
\end{align}

Así,

\begin{align*}
	\sin(2(\theta - \theta_{0})) & = \dfrac{1}{\varepsilon} - \dfrac{\alpha}{r^{2}\varepsilon}, \\
	\dfrac{\alpha}{r^{2}} = 1 + \varepsilon\sin(2(\theta - \theta_{0})).
\end{align*}

Como queremos que \(r\) sea el mínimo en \(\theta = 0\), elegimos
\(\theta_{0} = \tfrac{\pi}{4}\),

\begin{align*}
	\alpha & = r^{2} + \varepsilon r^{2}\cos(2\theta),                         \\
	\alpha & = r^{2} + \varepsilon r^{2}(\cos^{2}(\theta) - \sin^{2}(\theta)).
\end{align*}

Escribiendo la expresión anterior en coordenadas cartesianas
para determinar la ec. de la cónica que describe a la órbita, i.e.

\begin{align}
	\alpha        & = x^{2} + y^{2} + \varepsilon(x^{2} - y^{2}),                                   \\
	\Aboxedmain{1 & = \dfrac{1 + \varepsilon}{\alpha}x^{2} + \dfrac{1 - \varepsilon}{\alpha}y^{2},}
\end{align}

donde \(\varepsilon > 0, \alpha > 0\). Esta ecuación corresponde a
una hipérbola.
\end{document}
