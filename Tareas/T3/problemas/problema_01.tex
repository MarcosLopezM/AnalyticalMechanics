\documentclass[../main.tex]{subfiles}

\begin{document}
\begin{problema}
	Una partícula se mueve en una órbita circular sujeta a la
	fuerza central

	\begin{equation}
		F(r) = -\dfrac{k}{r^{2}}.
	\end{equation}

	Muestra que si \(k\) decrece a la mitad de su valor original,
	entonces la órbita de la partícula se vuelve parabólica.

	\startsolution

	Sabemos que las órbitas parabólicas surgen cuando \(E = 0\), por lo cual nos gustaría
	analizar la conservación de la energía. Primero calculamos el potencial a partir de
	la fuerza,

	\begin{align}
		U & = - \int \odif[sep-end=\medspace]{r} \biggl(\dfrac{-k}{r^{2}}\biggr),\nonumber \\
		U & = - \dfrac{k}{r}.\label{eq:PotentialU-P1}
	\end{align}

	Iremos construyendo la expresión para la energía total del sistema, comenzando por el
	potencial efectivo,

	\begin{equation}
		U_{eff}(r) = \dfrac{L^{2}}{2mr^{2}} - \dfrac{k}{r}.
		\label{eq:EffectivePotential-P1}
	\end{equation}

	Por otro lado, sabemos que para las fuerzas que se pueden obtener a partir de un potencial,
	podemos escribir la energía cinética, para este caso en particular, como:

	\begin{align}
		T & = -\dfrac{1}{2}U,\nonumber             \\
		T & = \dfrac{k}{2r}.\label{eq:KineticE-P1}
	\end{align}

	Ahora, obtenemos el valor de \(r\) para el que la energía es mínima, i.e.

	\begin{align*}
		\pdv{U_{eff}}{r}                                    & = -\dfrac{L^{2}}{mr^{3}} + \dfrac{k}{r^{2}} = 0 , \\
		\Biggl(kr - \dfrac{L^{2}}{m}\Biggr)\dfrac{1}{r^{3}} & = 0,
	\end{align*}

	tal que \(r_{1} = \infty\) y

	\begin{align}
		kr_{0} - \dfrac{L^{2}}{m} & = 0 ,                                      \\
		r_{0}                     & = \dfrac{L^{2}}{mk}.\label{eq:RMinimum-P1}
	\end{align}

	Sustituyendo la \zcref{eq:RMinimum-P1} en la \zcref{eq:EffectivePotential-P1},

	\begin{align*}
		U_{eff}(r_{0})            & = \dfrac{L^{2}}{2m \biggl(\tfrac{L^{2}}{mk}\biggr)^{2}} - \dfrac{k}{\biggl(\dfrac{L^{2}}{mk}\biggr)}, \\
		                          & = \dfrac{m^{2}k^{2}L^{2}}{2mL^{4}} - \dfrac{mk^{2}}{L^{2}},                                           \\
		                          & = \dfrac{mk^{2}}{2L^{2}} - \dfrac{mk^{2}}{L^{2}},                                                     \\
		                          & = -\dfrac{mk^{2}}{2L^{2}},                                                                            \\
		\Aboxedsec{U_{eff}(r_{0}) & = -\dfrac{k}{2r_{0}}.}
	\end{align*}

	Entonces, ya no hace falta que \(k^{\prime} = \tfrac{k}{2}\), puesto que la energía total es

	\begin{align*}
		E             & = \dfrac{k}{2r_{0}} - \dfrac{k}{2r_{0}}, \\
		\Aboxedmain{E & = 0.}
	\end{align*}
\end{problema}
\end{document}
