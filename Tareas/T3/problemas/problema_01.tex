\documentclass[../main.tex]{subfiles}

\begin{document}
\begin{problema}
	Una partícula se mueve en una órbita circular sujeta a la
	fuerza central

	\begin{equation}
		F(r) = -\dfrac{k}{r^{2}}.
	\end{equation}

	Muestra que si \(k\) decrece a la mitad de su valor original,
	entonces la órbita de la partícula se vuelve parabólica.
\end{problema}
\end{document}
