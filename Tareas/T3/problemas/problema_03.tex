\documentclass[../main.tex]{subfiles}

\begin{document}
\begin{problema}
	Encuentra la fuerza central responsable de que una partícula se
	mueva en una órbita dada por

	\begin{equation}
		r(\theta) = k\theta^{2}
	\end{equation}

	con \(k\) constante.
\end{problema}

\startsolution

Usando la ecuación de Binet podemos determinar la fuerza,

\begin{align*}
	\odv*{\biggl(\dfrac{1}{r}\biggr)}{\theta} & = -\dfrac{2}{k\theta^{3}},                                                                                                           \\
	\odv*{\biggl(\dfrac{1}{r}\biggr)}{\theta} & = \odv*{\biggl(-\dfrac{2}{k\theta^{3}}\biggr)}{\theta} = \dfrac{6}{k\theta^{4}} = \dfrac{6k}{(k\theta^{2})^{2}} = \dfrac{6k}{r^{2}}.
\end{align*}

Entonces,

\begin{align*}
	\dfrac{6k}{r^{2}} + \dfrac{1}{r} & = -\dfrac{mr^{2}}{L^{2}}F(r),                                            \\
	\Aboxedmain{F(r)                 & = \dfrac{-L^{2}}{mr^{2}}\Biggl(\dfrac{6k}{r^{2}} + \dfrac{1}{r}\Biggr).}
\end{align*}
\end{document}
