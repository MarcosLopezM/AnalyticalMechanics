\documentclass[../main.tex]{subfiles}

\begin{document}
\begin{problema}
	Use the Lagrange multiplier method to solve the following problem: A
	particle in a uniform gravitational field is free to move without
	friction on a paraboloid of revolution whose symmetry axis is vertical
	(opening upward). Obtain the force of constraint. Prove that for a
	given energy and angular momentum about the symmetry axis there are
	a minimum and maximum height to which the particle will go.

	\startsolution

	Por el problema 4 de la tarea anterior sabemos que el Lagrangiano del
	sistema es

	\begin{equation}
		L = \dfrac{1}{2}m \dot{r}^{2} + \dfrac{1}{2}m \dot{z}^{2} + \dfrac{1}{2}mr^{2} \dot{\theta}^{2}
		- mgz.
		\label{eq:lagrangian-p3}
	\end{equation}

	La constricción se debe a que el movimiento está restringido a la superficie del
	paraboloide, cuya ecuación es

	\begin{equation*}
		f(r, z) = z - cr^{2} = 0.
	\end{equation*}

	De esta manera, tenemos que la EOM para \(r\) es

	\begin{align}
		mr \dot{\theta}^{2} - m \ddot{r} - 2\lambda cr         & = 0,\nonumber \\
		\ddot{r} - r \dot{\theta}^{2} + \dfrac{2\lambda cr}{m} & = 0.
		\label{eq:eom-r}
	\end{align}

	Para \(\theta\),

	\begin{equation*}
		\odv{(mr^{2}\dot{\theta})}{t} = 0
	\end{equation*}

	Por lo que tenemos una cantidad conservada, i.e. el momento angular se conserva.

	\begin{equation}
		\dot{\theta} = \dfrac{L}{mr^{2}}.
		\label{eq:conservation-angular-momentum}
	\end{equation}

	La EOM para \(z\) es

	\begin{equation}
		-mg - m \ddot{z} + \lambda = 0.
		\label{eq:eom-z-p3}
	\end{equation}

	Resolviendo la \zcref{eq:eom-z-p3} para \(\lambda\),

	\begin{equation*}
		\lambda = m(\ddot{z} + g),
	\end{equation*}

	pero por la constricción sabemos que \(\ddot{z} = 2c(\dot{r}^{2} + r \ddot{r})\),
	entonces

	\begin{equation}
		\lambda = m \Bigl[2c(\dot{r}^{2} + r \ddot{r}) + g\Bigr].
		\label{eq:lagrange-multiplier}
	\end{equation}

	Ahora sustituimos la \zcref{eq:conservation-angular-momentum} en
	la \zcref{eq:eom-r} y resolviendo para \(\ddot{r}\),

	\begin{align*}
		\ddot{r} - r \Biggl(\dfrac{L^{2}}{m^{2}r^{4}}\Biggr) + \dfrac{2\lambda cr}{m} & = 0,                                                  \\
		\ddot{r}                                                                      & = \dfrac{L^{2}}{m^{2}r^{3}} - \dfrac{2\lambda cr}{m}.
	\end{align*}

	Por lo que \(\lambda\) queda como:

	\begin{align*}
		\lambda                  & = m \Biggl[2c \Biggl(\dot{r}^{2} + r \Biggl(\dfrac{L^{2}}{m^{2}r^{3}} - \dfrac{2\lambda cr}{m}\Biggr)\Biggr) + g\Biggr], \\
		                         & = m \Biggl[2c \Biggl(\dot{r}^{2} + \dfrac{L^{2}}{m^{2}r^{2}} - \dfrac{2\lambda cr^{2}}{m}\Biggr) + g\Biggr],             \\
		                         & = 2mc \dot{r}^{2} + \dfrac{2L^{2}c}{mr^{2}} - 4\lambda c^{2}r^{2} + mg,                                                  \\
		(1 + 4c^{2}r^{2})\lambda & = 2mc \dot{r}^{2} + \dfrac{2L^{2}c}{mr^{2}} + mg,                                                                        \\
		\Aboxedmain{\lambda      & = \dfrac{2m^{2}cr^{2} \dot{r}^{2} + 2L^{2}c + m^{2}r^{2}g}{mr^{2}(1 + 4c^{2}r^{2})}.}
	\end{align*}

	Ahora, para determinar la altura máxima y mínima, partimos de la conservación
	de la energía, \(E = T + U\), i.e.

	\begin{equation*}
		E = \dfrac{1}{2}m \dot{r}^{2} + \dfrac{1}{2}m \dot{z}^{2} + \dfrac{1}{2}mr^{2} \dot{\theta}^{2} + mgz.
	\end{equation*}

	Escribiendo todo en términos de \(z\) usando las constricción y la conservación
	del momento angular, tenemos que

	\begin{align*}
		\dot{z}          & = 2cr \dot{r},                                                                     \\
		\dot{z}^{2}      & = 4c^{2}r^{2} \dot{r}^{2} \implies \dot{r}^{2} = \dfrac{\dot{z}^{2}}{4c^{2}r^{2}}, \\
		\dot{\theta}     & = \dfrac{L}{mr^{2}},                                                               \\
		\dot{\theta}^{2} & = \dfrac{L^{2}}{m^{2}r^{4}}.
	\end{align*}

	Así,

	\begin{align*}
		E & = \dfrac{1}{2}m \Biggl(\dfrac{\dot{z}^{2}}{4c^{2}r^{2}}\Biggr) + \dfrac{1}{2}m \dot{z}^{2}
		+ \dfrac{1}{2}mr^{2} \Biggl(\dfrac{L^{2}}{m^{2}r^{4}}\Biggr) + mgz,                            \\
		E & = \dfrac{1}{2}m \Biggl(1 + \dfrac{1}{4cz}\Biggr)\dot{z}^{2} + \dfrac{cL^{2}}{2mz} + mgz.
	\end{align*}

	Recordamos que los puntos de retorno se obtienen cuando \(\dot{z} = 0\), por lo que
	la expresión para la energía se reduce a

	\begin{equation*}
		E = \dfrac{cL^{2}}{2mz} + mgz.
	\end{equation*}

	Encontrando las raíces,

	\begin{equation*}
		z = \dfrac{E \pm \sqrt{E^{2} - 2gcL^{2}}}{2mg}.
	\end{equation*}

	De la expresión anterior tenemos que el máximo de \(z\) se alcanza cuando \(L = 0\), i.e.

	\begin{align*}
		z_{\text{máx}}             & = \dfrac{E + \sqrt{E^{2}}}{2mg}, \\
		                           & = \dfrac{2E}{2mg},               \\
		\Aboxedmain{z_{\text{máx}} & = \dfrac{E}{mg}.}
	\end{align*}

	Mientras que el mínimos es

	\begin{align*}
		z_{\text{mín}}             & = \dfrac{E - E}{2mg}, \\
		\Aboxedmain{z_{\text{mín}} & = 0.}
	\end{align*}
\end{problema}
\end{document}
