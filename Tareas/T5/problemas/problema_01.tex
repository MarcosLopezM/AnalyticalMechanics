\documentclass[../main.tex]{subfiles}

\begin{document}
\begin{problema}
	A particle moves in one dimension, in a potential \(V(x)\), where \(x\)
	is the spatial coordinate. The dynamics is governed by the Lagrangian

	\begin{equation*}
		L = \dfrac{1}{2}m^{2}\dot{x}^{4} + m \dot{x}^{2}V(x) - V(x)^{2}.
	\end{equation*}

	Show that the resulting equation of motion is identical to that which
	arises from the more traditional Lagrangian, \(L = \tfrac{1}{2}m \dot{x}^{2} - V(x)\).

	\startsolution

	Calculamos los elementos de las ecuaciones de Euler-Lagrange

	\begin{align*}
		\pdv{L}{\dot{x}}                       & = \dfrac{4}{12}m^{2}\dot{x}^{3} + 2m \dot{x}V(x),                                   \\
		\odv*{\Bigl(\pdv{L}{\dot{x}}\Bigr)}{t} & = m^{2}\dot{x}^{2}\ddot{x} + 2m \Biggl(\ddot{x}V(x) + \dot{x}^{2}\pdv{V}{x}\Biggr), \\
		\pdv{L}{x}                             & = m \dot{x}^{2}\pdv{V}{x} - 2V \pdv{V}{x}.
	\end{align*}

	Así,

	\begin{align*}
		m^{2} \dot{x}^{2}\ddot{x} + 2m \Biggl(\ddot{x}V(x) + \dot{x}^{2}\pdv{V}{x}\Biggr) -
		\Biggl(m \dot{x}^{2}\pdv{V}{x} - 2V \pdv{V}{x}\Biggr) = 0,                               \\
		m^{2}\dot{x}^{2}\ddot{x} + 2m \ddot{x}V + 2m \dot{x}^{2}\pdv{V}{x} 0 m \dot{x}^{2}\pdv{V}{x}
		+ 2V \pdv{V}{x}                                                                   & = 0, \\
		(m \dot{x}^{2} + 2V)m \ddot{x} + \pdv{V}{x}(m \dot{x}^{2} + 2V)                   & = 0, \\
		2\Biggl(m \ddot{x} + \pdv{V}{x}\Biggr)\Biggl(\dfrac{1}{2}m \dot{x}^{2} + V\Biggr) & = 0, \\
		\Biggl(m \ddot{x} + \pdv{V}{x}\Biggr)\Biggl(\dfrac{1}{2}m \dot{x}^{2} + V\Biggr)  & = 0.
	\end{align*}

	De esta manera tenemos que si la energía cinética no es igual a la energía
	potencial recuperamos la segunda ley de Newton, i.e.

	\begin{empheq}[box = \mainresult]{equation*}
		m \ddot{x} = -\pdv{V}{x}.
	\end{empheq}

	Que es la misma ecuación de movimiento que se obtiene del Lagrangiano
	tradicional,

	\begin{align*}
		\odv*{\Biggl(\pdv{L}{\dot{x}}\Biggr)}{t} - \pdv{L}{x} & = 0,            \\
		m \ddot{x} + \pdv{V}{x}                               & = 0,            \\
		\Aboxedmain{m \ddot{x}                                & = -\pdv{V}{x}.}
	\end{align*}
\end{problema}
\end{document}
