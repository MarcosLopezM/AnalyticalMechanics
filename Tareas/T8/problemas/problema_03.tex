\documentclass[../main.tex]{subfiles}

\begin{document}
\begin{problema}
	A straight homogeneous rod \(OA\) (length \(2l\) and mass \(m\)) can
	rotate freely about the fixed end point \(O\). Initially the rod is
	horizontal (\(\theta = 0\)) and rotates about the vertical direction through
	\(O\) with angular velocity \(\dot{\phi} = \omega\). Under the influence
	of gravity it starts to rotate about a horizontal axis through \(O\).
	Calculate \(\dot{\phi}\) as a function of \(\theta\) in the ensuing
	motion and calculate the turning points in the \(\theta\)-motion.

	\begin{figure}[htb]
		\centering
		\includegraphics[width=0.48\textwidth]{T8-p-3.png}
	\end{figure}
\end{problema}
\end{document}
