\documentclass[../main.tex]{subfiles}

\begin{document}
\begin{problema}
	The relativistic Lagrangian for a particle of a rest mass \(m_{0}\)
	moving along the \(x\)-axis under the potential field \(V(x)\)
	is given by

	\begin{equation*}
		L = m_{0}c^{2} \Biggl(1 - \Biggl(1 - \dfrac{\dot{x}^{2}}{c^{2}}\Biggr)^{1/2}\Biggr) - V(x).
	\end{equation*}

	Show that the corresponding Hamiltonian is given by

	\begin{equation*}
		H = m_{0}c^{2} \Biggl(1 + \dfrac{p_{x}^{2}}{m_{0}^{2}c^{2}}\Biggr)^{1/2} - m_{0}c^{2} + V(x),
	\end{equation*}

	where \(p_{x}\) is the generalized momentum conjugate to \(x\).

	\startsolution

	Calculamos el Hamiltoniano dado por

	\begin{equation*}
		H = p_{i}\dot{q}_{i} - L.
	\end{equation*}

	El momento generalizado es:

	\begin{align*}
		p_{x} & = \pdv{L}{\dot{x}} = m_{0}c^{2}\Biggl[-\dfrac{1}{2}\Bigl(1 - \dfrac{\dot{x}^{2}}{c^{2}}\Bigr)^{-1/2}\Bigl(-\dfrac{2 \dot{x}}{c^{2}}\Bigr)\Biggr], \\
		p_{x} & = \dfrac{m_{0} \dot{x}}{\Bigl(1 - \dfrac{\dot{x}^{2}}{c^{2}}\Bigr)^{1/2}}.
	\end{align*}

	Resolviendo la expresión anterior para \(\dot{x}\),

	\begin{align}
		p_{x}\Bigl(1 - \dfrac{\dot{x}^{2}}{c^{2}}\Bigr)^{1/2}                        & = m_{0} \dot{x},\nonumber                                                   \\
		p_{x}^{2}\Bigl(1 - \dfrac{\dot{x}^{2}}{c^{2}}\Bigr)                          & = m_{0}^{2} \dot{x}^{2},\nonumber                                           \\
		\dfrac{p_{x}^{2}}{m_{0}^{2}} - \dfrac{p_{x}^{2} \dot{x}^{2}}{m_{0}^{2}c^{2}} & = \dot{x}^{2},\nonumber                                                     \\
		\Biggl(1 + \dfrac{p_{x}^{2}}{m_{0}^{2}c^{2}}\Biggr) \dot{x}^{2}              & = \dfrac{p_{x}^{2}}{m_{0}^{2}},\nonumber                                    \\
		\Aboxedmain{\dot{x}                                                          & = \dfrac{p_{x}}{m_{0}\Biggl(1 + \dfrac{p_{x}^{2}}{m_{0}^{2}c^{2}}\Biggr)}.}
		\label{eq:x-derivative}
	\end{align}

	Por lo que el Hamiltoniano es

	\begin{align*}
		H & = p_{x} \dot{x} - L,                                                                                                 \\
		  & = p_{x} \dot{x} + m_{0}c^{2}\Biggl[\Biggl(1 - \Bigl(1 - \dfrac{\dot{x}^{2}}{c^{2}}\Bigr)^{1/2}\Biggr) - V(x)\Biggr], \\
		  & = p_{x} \dot{x} - m_{0}c^{2} + m_{0}c^{2}\Bigl(1 - \dfrac{\dot{x}^{2}}{c^{2}}\Bigr)^{1/2} + V(x).
	\end{align*}

	Sustituyendo la \zcref{eq:x-derivative} y simplificando,

	\begin{align*}
		H & = p_{x}\Biggl(\dfrac{p_{x}}{m_{0}\Bigl(1 + \tfrac{p_{x}^{2}}{m_{0}^{2}c^{2}}\Bigr)^{1/2}}\Biggr)
		- m_{0}c^{2} + m_{0}c^{2}\Biggl[1 - \dfrac{1}{c^{2}}\Biggl(\dfrac{p_{x}^{2}}{m_{0}^{2}\Bigl(1 + \tfrac{p_{x}^{2}}{m_{0}^{2}c^{2}}\Bigr)}\Biggr)\Biggr]^{1/2} + V(x), \\
		  & = \dfrac{p_{x}^{2}}{m_{0}\Bigl(1 + \tfrac{p_{x}^{2}}{m_{0}^{2}}\Bigr)^{1/2}} -
		m_{0}c^{2} +
		m_{0}c^{2}\Biggl[\dfrac{m_{0}^{2}c^{2} + p_{x}^{2} - p_{x}^{2}}{m_{0}^{2}c^{2} \Bigl(1 + \dfrac{p_{x}^{2}}{m_{0}^{2}c^{2}}\Bigr)}\Biggr]^{1/2}
		+ V(x),                                                                                                                                                              \\
		  & = \dfrac{p_{x}^{2}}{m_{0}\Bigl(1 + \tfrac{p_{x}^{2}}{m_{0}^{2}}\Bigr)^{1/2}} -
		m_{0}c^{2}+ m_{0}c^{2}\Biggl(\dfrac{m_{0}c}{m_{0}c \Bigl(1 + \dfrac{p_{x}^{2}}{m_{0}^{2}c^{2}}\Bigr)^{1/2}}\Biggr)
		+ V(x),                                                                                                                                                              \\
		  & =	\dfrac{\tfrac{p_{x}^{2}}{m_{0}} + m_{0}c^{2}}{\Bigl(1 + \tfrac{p_{x}^{2}}{m_{0}^{2}c^{2}}\Bigr)} -
		m_{0}c^{2} + V(x),                                                                                                                                                   \\
		  & = \dfrac{m_{0}c^{2}\Bigl(1 + \tfrac{p_{x}^{2}}{m_{0}^{2}c^{2}}\Bigr)}{\Bigl(1 + \tfrac{p_{x}^{2}}{m_{0}^{2}c^{2}}\Bigr)} - m_{0}c^{2} + V(x).
	\end{align*}

	Por lo tanto, el Hamiltoniano es

	\begin{empheq}[box = \mainresult]{equation*}
		H = m_{0}c^{2}\Bigl(1 + \tfrac{p_{x}^{2}}{m_{0}^{2}c^{2}}\Bigr) -
		m_{0}c^{2} + V(x).
	\end{empheq}
\end{problema}
\end{document}
