\documentclass[../main.tex]{subfiles}

\begin{document}
\begin{problema}
	A particle in a uniform gravitational field is constrained to the surface
	of a sphere, centered at the origin, with a radius \(r(t)\) which is a
	given function of time. Obtain the Hamiltonian and the canonical equations.
	Discuss energy conservation. Is the Hamiltonian the total energy?

	\startsolution

	Como el movimiento está constriñido a una esfera, usamos coordenadas
	para describir el movimiento de la partícula, por lo que la energía cinética
	es

	\begin{equation*}
		T = \dfrac{1}{2}m\Bigl(\dot{r}^{2} + r^{2} \dot{\theta}^{2} + r^{2}\sin^{2}\theta \phi^{2}\Bigr),
	\end{equation*}

	Mientras que la energía potencial es

	\begin{equation*}
		U = mgr\cos\theta.
	\end{equation*}

	De esta forma, el Lagrangiano del sistema es

	\begin{equation*}
		L = \dfrac{1}{2}m \Bigl(\dot{r}^{2} + r^{2} \dot{\theta}^{2} + r^{2}\sin^{2}\theta \dot{\phi}^{2}\Bigr)
		- mgr\cos\theta.
	\end{equation*}

	Pero como \(r(t)\) se nos da explícitamente, i.e. no es una variable dinámica.
	El problema se reduce a uno con 2 DOF.

	Calculamos los momentos conjugados

	\begin{alignat*}{4}
		p_{\theta}   & = \pdv{L}{\dot{\theta}} & {}={}                                   & mr^{2} \dot{\theta}             & {}\Longrightarrow{} &
		\dot{\theta} & {}={}                   & \dfrac{p_{\theta}}{mr^{2}},                                                                       \\
		p_{\phi}     & = \pdv{l}{\dot{\phi}}   & {}={}                                   & mr^{2}\sin^{2}\theta \dot{\phi} & {}\Longrightarrow{} &
		\dot{\phi}   & {}={}                   & \dfrac{p_{\phi}}{mr^{2}\sin^{2}\theta}.
	\end{alignat*}

	El Hamiltoniano del sistema es

	\begin{align*}
		H             & = p_{\theta} \dot{\theta} + p_{\phi} \dot{\phi} - \dfrac{1}{2}m\Bigl(\dot{r}^{2} + r^{2} \dot{\theta}^{2} + r^{2}\sin^{2}\theta \dot{\phi}^{2}\Bigr)
		+ mgr\cos\theta,\nonumber                                                                                                                                                      \\
		              & = p_{\theta}\Bigl(\dfrac{p_{\theta}}{mr^{2}}\Bigr) +
		p_{\phi}\Bigl(\dfrac{p_{\phi}}{mr^{2}\sin^{2}\theta}\Bigr)
		- \dfrac{1}{2}mr^{2}\Bigl(\dfrac{p_{\theta}}{mr^{2}}\Bigr)^{2}
		- \dfrac{1}{2}mr^{2}\sin^{2}\theta \Bigl(\dfrac{p_{\phi}}{mr^{2}\sin^{2}\theta}\Bigr)^{2}
		- \dfrac{1}{2}m \dot{r}^{2}
		+ mgr\cos\theta,                                                                                                                                                     \nonumber \\
		\Aboxedmain{H & = \dfrac{p_{\theta}^{2}}{2mr^{2}} + \dfrac{p_{\phi}^{2}}{2mr^{2}\sin^{2}\theta} - \dfrac{1}{2}m \dot{r}^{2} + mgr\cos\theta.}
	\end{align*}

	Por lo que las ecuaciones de Hamilton son

	\begin{alignat*}{4}
		\text{Para } \theta & \colon \quad \dot{\theta} & {}={}              & \dfrac{p_{\theta}}{mr^{2}} \quad
		                    & {}\wedge{}                & \quad - p_{\theta} & {}={}                                        & \dfrac{p_{\phi}}{4mr^{2}\sin\theta\cos\theta} - mgr\sin\theta, \\
		\text{Para } \phi   & \colon \quad \dot{\phi}   & {}={}              & \dfrac{p_{\phi}}{mr^{2}\sin^{2}\theta} \quad
		                    & {}\wedge{}                & \quad p_{\phi}     & {}={}                                        & 0.
	\end{alignat*}

	Por otro lado, como el Lagrangiano no depende explícitamente del tiempo, \(E = T + U\),

	\begin{equation*}
		E  = \dfrac{p_{\theta}^{2}}{2mr^{2}} + \dfrac{p_{\phi}^{2}}{2mr^{2}\sin^{2}\theta} +
		\dfrac{1}{2}m \dot{r}^{2}
		+ mgr\cos\theta.
	\end{equation*}

	¿Es \(H = E\)? La respuesta es no, pues

	\begin{equation}
		E = H + m \dot{r}^{2}.
		\label{eq:energy-system}
	\end{equation}

	Es decir, el Hamiltoniano no corresponde a la energía total
	del sistema, pues aunque el potencial es independiente de las velocidades y
	no depende explícitamente del tiempo, la transformación de coordenadas sí depende
	explícitamente del tiempo.

	Finalmente, para determinar si la energía del sistema se conserva,
	calculamos la derivada respecto al tiempo de la \zcref{eq:energy-system},

	\begin{align*}
		\odv{E}{t} & = \odv{H}{t} + 2m \dot{r} \ddot{r},\nonumber \\
		           & = \pdv{H}{t} + 2m \dot{r} \ddot{r},          \\
		           & = m \dot{r} \ddot{r} + 2m \dot{r} \ddot{r},  \\
		           & = 3m \dot{r}\ddot{r},                        \\
		           & \neq 0.
	\end{align*}

	Por lo tanto, la energía del sistema no se conserva.
\end{problema}
\end{document}
