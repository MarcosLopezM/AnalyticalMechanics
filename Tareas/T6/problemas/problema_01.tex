\documentclass[../main.tex]{subfiles}

\begin{document}
\begin{problema}
	It has been previously noted that the total time derivative of a function
	of \(q_{i}\) and \(t\) can be added to the Lagrangian without changing the
	equations of motion (tarea 4). What does such an addition do to the canonical
	momenta and the Hamiltonian? Show that the equations of motion in terms of
	the new Hamiltonian reduce to the original Hamilton's equations of motion.

	\startsolution

	Recordamos la forma del Lagrangiano \(L^{\prime}\),

	\begin{equation*}
		L^{\prime} = L + \odv{F(q_{i}, t)}{t}.
	\end{equation*}

	Ahora, calculamos el momento generalizado,

	\begin{equation}
		p_{i}^{\prime} = \pdv{L^{\prime}}{\dot{q}_{i}} = \pdv{L}{\dot{q}_{i}} +
		\pdv*{\Biggl(\pdv{F}{t}\Biggr)}{\dot{q}_{i}}.
	\end{equation}

	Recordando que el último término se puede escribir como

	\begin{equation}
		\odv{F(q_{i}, t)}{t} = \odv{F(q_{i}, t)}{q_{i}} \dot{q}_{i} + \pdv{F}{t}.
		\label{eq:dt-F}
	\end{equation}

	Así, el momento generalizado queda como

	\begin{align*}
		p_{i}^{\prime} & = \pdv{L}{\dot{q}_{i}} + \pdv*{\Biggl[\pdv{F(q_{i}, t)}{q_{i}} \dot{q}_{i} + \pdv{F}{t}\Biggr]}{\dot{q}_{i}},\nonumber \\
		p_{i}^{\prime} = \pdv{L}{\dot{q}_{i}} + \pdv{F}{q_{i}},
	\end{align*}

	pero \(p_{i} = \pdv{p_{i}}{q_{i}}\),

	\begin{empheq}[box = \mainresult]{equation}
		p_{i}^{\prime} = p_{i} + \pdv{F}{q_{i}}.
		\label{eq:prime-conjugate-momenta}
	\end{empheq}

	Entonces, el Hamiltoniano es

	\begin{align*}
		H^{\prime} & = p_{i}^{\prime} \dot{q}_{i} - L^{\prime},                                        \\
		           & = \Biggl(p_{i} + \pdv{F}{q_{i}}\Biggr)\dot{q}_{i} - \Biggl(L + \odv{F}{t}\Biggr).
	\end{align*}

	Sustituyendo la \zcref{eq:dt-F}, tenemos que

	\begin{align}
		H^{\prime}             & = p_{i} \dot{q}_{i} + \pdv{F}{q_{i}} \dot{q}_{i} - L - \pdv{F}{q_{i}} \dot{q}_{i} - \pdv{F}{t},\nonumber \\
		                       & = p_{i} \dot{q}_{i} - L - \pdv{F}{t},\nonumber                                                           \\
		\Aboxedmain{H^{\prime} & = H - \pdv{F}{t}.}\label{eq:prime-hamiltonian}
	\end{align}

	Calculamos ahora las ecuaciones de Hamilton de \(H^{\prime}\), recordando que las
	ecuaciones para \(H\) son

	\begin{align*}
		\dot{q}_{i}   & = \pdv{H}{p_{i}}, \\
		- \dot{p}_{i} & = \pdv{H}{q_{i}}.
	\end{align*}

	Así, para la primera ecuación tenemos

	\begin{align*}
		\pdv{H^{\prime}(q_{i}, p_{i}^{\prime})}{p_{i}^{\prime}} & = \pdv*{\Biggl(p_{i}^{\prime} \dot{q}_{i} - L^{\prime}(q_{i}, \dot{q}_{i})\Biggr)}{p_{i}^{\prime}}, \\
		                                                        & = \pdv{(p_{i}^{\prime} \dot{q}_{i})}{p_{i}^{\prime}} - \pdv{L^{\prime}}{p_{i}^{\prime}},            \\
		                                                        & = \dot{q}_{i}.
	\end{align*}

	Por lo tanto,

	\begin{empheq}[box = \mainresult]{equation*}
		\dot{q}_{i} = \pdv{H^{\prime}}{p_{i}^{\prime}}.
	\end{empheq}

	Ahora, para la segunda ecuación, sustituimos \zcref{eq:prime-conjugate-momenta,eq:prime-hamiltonian},

	\begin{align*}
		-\Biggl(\dot{p}_{i} + \odv*{\Biggl(\pdv{F}{q_{i}}\Biggr)}{t}\Biggr) & =
		\pdv*{\Biggl(H(q_{i}, p_{i}) - \pdv{F}{t}\Biggr)}{q_{i}},               \\
		- \dot{p}_{i} - \odv*{\Biggl(\pdv{F}{q_{i}}\Biggr)}{t}              & =
		\pdv{H}{q_{i}} - \pdv*{\Biggl(\pdv{F}{t}\Biggr)}{q_{i}},
	\end{align*}

	pero \(\odv*{\Bigl(\pdv{F}{q_{i}}\Bigr)}{t} = \pdv*{\Bigl(\pdv{F}{t}\Bigr)}{q_{i}}\).
	Así,

	\begin{empheq}[box = \mainresult]{equation*}
		- \dot{p}_{i} = \pdv{H}{q_{i}}.
	\end{empheq}

	Por lo tanto, las ecuaciones de Hamilton de \(H^{\prime}\) se reducen
	a las ecuaciones de Hamilton para \(H\).
\end{problema}
\end{document}
