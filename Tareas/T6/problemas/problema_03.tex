\documentclass[../main.tex]{subfiles}

\begin{document}
\begin{problema}
	Consider a particle of mass \(m\) moving in two dimensions, subject
	to a force \(\vect{F} = -kx \uvec{x} + K \uvec{y}\), where \(k\)
	and \(K\) are positive constants. Write down the Hamiltonian and
	Hamilton's equations, using \(x\) and \(y\) as generalized
	coordinates. Solve the latter and describe the motion.

	\startsolution

	La energía cinética para este sistema es

	\begin{equation}
		T = \dfrac{1}{2}m(\dot{x}^{2} + \dot{y}^{2}).
		\label{eq:T-p3}
	\end{equation}

	Mientras que para obtener el potencial, primero verificamos si
	la fuerza es conservativa, i.e. \(\nabla \mul{\vect{F}} = 0\),

	\begin{align*}
		\nabla \mul{\vect{F}} & =
		\begin{vNiceMatrix}
			\uvec{i} & \uvec{j} & \uvec{k} \\
			\pdif{x} & \pdif{y} & \pdif{z} \\
			-kx      & k        & 0        \\
		\end{vNiceMatrix},                                                 \\
		                      & = \uvec{i}(0 - 0) - \uvec{j}(0 - 0) + \uvec{k}(0 - 0), \\
		                      & = 0.
	\end{align*}

	Por lo tanto, la fuerza es conservativa y podemos escribirla como

	\begin{equation*}
		\vect{F} = - \nabla{V}.
	\end{equation*}

	Así, tenemos que

	\begin{align}
		\pdv{V}{k} & = kx\label{eq:V_x}, \\
		\pdv{V}{y} & = -K.\label{eq:V_y}
	\end{align}

	De la \zcref{eq:V_y} tenemos que

	\begin{align*}
		V = \int \odif[sep-end=\medspace]{x} kx, \\
		V & = k \dfrac{x^{2}}{2} + C(y).
	\end{align*}

	Sustituyendo en la \zcref{eq:V_y},

	\begin{equation*}
		\pdv{V}{y} = \pdv{C}{y} = - K.
	\end{equation*}

	Así,

	\begin{equation*}
		C(y) = -Ky + \alpha,
	\end{equation*}

	donde podemos elegir \(\alpha = 0\) sin perdida de generalidad.

	Por lo tanto, la energía potencial es

	\begin{empheq}[box = \mainresult]{equation*}
		V = \dfrac{1}{2}kx^{2} - Ky.
	\end{empheq}

	Y notemos que la energía potencial no depende de la velocidad ni depende
	explícitamente del tiempo, por lo que \(H = T + V\). De esta manera,
	escribimos la \zcref{eq:T-p3} en términos de los momentos conjugados, i.e.

	\begin{equation*}
		T = \dfrac{p_{x}^{2}}{2m} + \dfrac{p_{y}^{2}}{2m}.
	\end{equation*}

	Así, el Hamiltoniano del sistema es

	\begin{equation*}
		H = \dfrac{p_{x}^{2}}{2m} + \dfrac{p_{y}^{2}}{2m} + k \dfrac{x^{2}}{2} - Ky.
	\end{equation*}

	Por lo que las ecuaciones de Hamilton son

	\begin{alignat*}{2}
		\dot{x} & = \dfrac{p_{x}}{m};\quad - \dot{p}_{x} & {}={} & kx, \\
		\dot{y} & = \dfrac{p_{y}}{m};\quad \dot{p}_{y}   & {}={} & K.
	\end{alignat*}

	Entonces, la EOM para x es

	\begin{equation*}
		\ddot{x} + \omega^{2}x = 0,
	\end{equation*}

	con \(\omega^{2} = \tfrac{k}{m}\). Y, para \(y\),

	\begin{equation}
		\ddot{y} = \dfrac{K}{m}.
	\end{equation}

	Resolviendo cada una de las EOM tenemos que

	\begin{equation*}
		x(t) = A \cos\bigl(\omega t\bigr) + B \sin\bigl(\omega t\bigr).
	\end{equation*}

	Y,

	\begin{align*}
		\dot{y} & = \dfrac{K}{m}t + v_{0y},              \\
		y(t)    & = y_{0} + v_{0y}t + \dfrac{K}{m}t^{2}.
	\end{align*}

	Finalmente, tenemos que el movimiento en la dirección \(x\) es un
	movimiento oscilatorio, mientras que en la dirección \(y\),
	tenemos un movimiento similar al de caída libre.
\end{problema}
\end{document}
