\documentclass[../main.tex]{subfiles}

\begin{document}
\begin{problema}
	Un niño está sentado en la parte superior de un montón
	de hielo semiesférico. Se le da un pequeño empujón y
	comienza a deslizarse hacia abajo por el hielo.
	(a) Demostrar que si el hielo no presenta fricción
	el niño pierde contacto con el hielo en un punto
	cuya altura es \(2R/3\) en donde \(R\) es el radio
	de la semiesfera.
	(b) Si existe fricción entre el niño y el hielo,
	¿se desprenderá de la superficie en un punto inferior
	o superior que el encontrado en (a)?
\end{problema}
\end{document}
