\documentclass[../main.tex]{subfiles}

\begin{document}
\begin{problema}
	Un niño está sentado en la parte superior de un montón
	de hielo semiesférico. Se le da un pequeño empujón y
	comienza a deslizarse hacia abajo por el hielo.
	(a) Demostrar que si el hielo no presenta fricción
	el niño pierde contacto con el hielo en un punto
	cuya altura es \(2R/3\) en donde \(R\) es el radio
	de la semiesfera.
	(b) Si existe fricción entre el niño y el hielo,
	¿se desprenderá de la superficie en un punto inferior
	o superior que el encontrado en (a)?

	\startsolution

	Las fuerzas que actúan sobre el niño son la fuerza debida a la aceleración
	de la gravedad, la fuerza normal \(F_{N}\) y la fuerza centrípeta \(F_{c}\). Por
	segunda ley de Newton, tenemos

	\begin{equation*}
		mg\cos\theta - F_{N}  = m \dfrac{v^{2}}{R}.
	\end{equation*}

	Pero como nos interesa encontrar el punto donde el niño se desprende,
	\(F_{N} = 0\), por lo que

	\begin{align}
		mg\cos\theta & = \dfrac{mv^{2}}{R}, \nonumber  \\
		g\cos\theta  & = \dfrac{v^{2}}{R},\nonumber    \\
		v^{2}        & = gR\cos\theta.\label{eq:vSqrd}
	\end{align}

	Ahora, para encontrar la altura nos fijamos en la conservación de la energía. Sabemos
	que el niño comienza en la parte superior del montón de hielo, por lo que tiene una
	energía potencial inicial \(mgR\), pero no energía cinética. Mientras que en el punto
	en el que se desprende la energía potencial es \(mgh\), con \(h = R\cos\theta\) y
	la energía cinética es \(\tfrac{1}{2}mv^{2}\). Por lo que,

	\begin{equation}
		mgR = \dfrac{1}{2}mv^{2} + mgR\cos\theta.
		\label{eq:energy}
	\end{equation}

	Sustituyendo la \zcref{eq:vSqrd} en la \zcref{eq:energy},

	\begin{align*}
		gR                    & = \dfrac{1}{2}(gR\cos\theta) + gR\cos\theta, \\
		1                     & = \biggl(\tfrac{1}{2} + 1\biggr)\cos\theta,  \\
		1                     & = \dfrac{3}{2}\cos\theta,                    \\
		\Aboxedsec{\cos\theta & = \dfrac{3}{2}.}
	\end{align*}

	De esta manera,

	\begin{empheq}[box=\mainresult]{equation*}
		h = \dfrac{2}{3}R.
	\end{empheq}

\end{problema}
\end{document}
