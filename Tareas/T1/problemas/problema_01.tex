\documentclass[../main.tex]{subfiles}

\begin{document}
\begin{problema}
	La posición de una partícula como función del tiempo \(t\) está dada
	por \(\vect{x}(t) = A(\uveci\cos(\omega t) \uvecj\sin(\omega t))\).
	Encuentra su velocidad \(\vect{v}\) y aceleración \(\vect{a}\)
	como función del tiempo y calcula sus magnitudes.
	¿Qué dimensiones tienen \(A\) y \(\omega\)?
\end{problema}
\end{document}
