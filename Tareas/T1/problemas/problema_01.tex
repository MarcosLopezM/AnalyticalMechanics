\documentclass[../main.tex]{subfiles}

\begin{document}
\begin{problema}
	La posición de una partícula como función del tiempo \(t\) está dada
	por \(\vect{x}(t) = A(\uveci\cos(\omega t) + \uvecj\sin(\omega t))\).
	Encuentra su velocidad \(\vect{v}\) y aceleración \(\vect{a}\)
	como función del tiempo y calcula sus magnitudes.
	¿Qué dimensiones tienen \(A\) y \(\omega\)?

	\startsolution

	Para encontrar la velocidad calculamos \(\dot{x}\),

	\begin{align}
		\va{v} = \dot{x}      & = \odv*{\Biggl[A(\uveci\cos\omega t \uvecj\sin\omega t)\Biggr]}{t},\nonumber       \\
		                      & = A\odv*{\biggl[\uveci\cos\omega t + \uvecj\sin\omega t\biggr]}{t},\nonumber       \\
		                      & = A \biggl[\uveci(-\omega\sin\omega t) + \uvecj\omega\cos\omega t\biggr],\nonumber \\
		\Aboxedmain{\va{v}(t) & = -A\omega\sin\omega t \uveci + A\omega\cos\omega t \uvecj.}\label{eq:Velocity}
	\end{align}


	Para encontrar la aceleración, calculamos \(\ddot{x}\) o \(\dot{\va{v}}\), entonces

	\begin{align*}
		\va{a}(t)             & = \odv*{\Biggl[A\omega \biggl(-\sin\omega t \uveci + \cos\omega t \uvecj\biggr)\Biggr]}{t},\nonumber \\
		                      & = A\omega \Biggl[-\omega\cos\omega t \uveci - \omega\sin\omega t \uvecj\Biggr],\nonumber             \\
		\Aboxedmain{\va{a}(t) & = -A\omega^{2}\Biggl[\cos\omega t \uveci + \sin\omega t \uvecj\Biggr].}
	\end{align*}

	Hacemos un análisis dimensional para encontrar las dimensiones de \(A\) y \(\omega\).
	Recordemos que

	\begin{equation*}
		[\va{x}] = [L],\qquad [\va{v}] = \Biggl[\dfrac{L}{T}\Biggr],\qquad [\va{a}] = \Biggl[\dfrac{L}{T^{2}}\Biggr].
	\end{equation*}

	Entonces, ¿qué dimensiones debe tener \(A\) y \(\omega\) para que esto se cumpla?
	Sabemos que las dimensiones de las cantidades físicas están dada por la amplitud
	\(A\), por lo que el resto de la expresión debe ser adimensional. Es decir, las
	dimensiones de \([t] = [T]\), por lo que

	\begin{empheq}[box=\mainresult]{equation*}
		[\omega] = \biggl[\dfrac{1}{T}\biggr].
	\end{empheq}

	Entonces,

	\begin{empheq}[box=\mainresult]{equation*}
		[A] = [L].
	\end{empheq}
\end{problema}
\end{document}
